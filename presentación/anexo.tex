%!TeX spellcheck = es_ES
%!TEX program = lualatex
\documentclass[hyperref={unicode}]{beamer}

\usepackage[spanish, es-tabla, es-nodecimaldot]{babel}
\usepackage{graphicx}
\usepackage[unicode]{hyperref}
\usepackage{bookmark}
\usepackage{tikz}
\usepackage{etoolbox}
\usepackage{subfig}
\usepackage{booktabs}
\usepackage{amsmath}
\usepackage{amsfonts}
\usepackage{amssymb}
\usetikzlibrary{spy}


\usepackage{pgf}
\usepackage{tikz}
\usetikzlibrary{arrows,automata,fit}

\newcommand{\ejemplografo}{
\begin{tikzpicture}[>=stealth',shorten >=1pt,auto,node distance=3cm,semithick]
  \tikzstyle{every state}=[draw=black,text=black]

  \node[state]         (A)                    {$A$};
  \node[state]         (B) [right of=A]       {$B$};
  \node[state]         (C) [below of=A]       {$C$};
  \node[state]         (D) [right of=C]       {$D$};
  
  \path (A) edge  [bend left]   node {}  (B)
            edge  [bend right]  node {}  (C)
        (B) edge                node {}  (C)
        (C) edge  [bend right]  node {}  (D);
\end{tikzpicture}}

\newtheorem{teo}{\textbf{\color{ExecusharesBlue}Teorema}}
\newtheorem{defi}{\textbf{\color{ExecusharesBlue}Definición}}

\usetheme{TFM}

\title{Anexo}
\subtitle{Trabajo Fin de Máster \\ \textbf{Máster en Ingeniería de Sistemas de Decisión} \\ \textit{Curso 2016--2017}}
\author{José Ignacio Escribano Pablos}
\institute{\begin{tabular}{c}
Ana Elizabeth García Sipols \\
Miguel Romance del Río     
\end{tabular}}
\titlegraphic{\includegraphics[width=2cm]{imagenes/urjc.png}}
\date{20 de enero de 2017}

\setcounter{tocdepth}{2}

\makeatletter
\patchcmd{\beamer@sectionintoc}
{\vfill}
{\vskip\itemsep}
{}
{}
\makeatother  


\setcounter{showSlideNumbers}{1}

\begin{document}
\setcounter{showProgressBar}{0}
\setcounter{showSlideNumbers}{0}

\frame{\titlepage}

\begin{frame}{Regresión lineal}
	La regresión lineal viene dada por el modelo
	
	\begin{equation}
	y_i = \alpha_1 x_{i1} + \dots + \alpha_p x_{ip} + \alpha_0, \quad i = 1,...,n
	\end{equation}
	
	donde $y_i$ es la variable salida, $x_{i}$ es la variable de entrada, $n$ es el número de observaciones y $p$ es el número de variables de entrada.
	
	\begin{figure}
		\centering
		\subfloat[Dos dimensiones]{\includegraphics[width=0.3\textwidth]{imagenes/linear_regression2d.png}}\qquad\qquad
		\subfloat[Tres dimensiones]{\includegraphics[width=0.3\textwidth]{imagenes/linear_regression.png}}
		
	\end{figure}
\end{frame}


\begin{frame}{Regresión lineal (bidimensional)}
	En particular, en el caso bidimensional el modelo viene dado por
	
	\begin{equation}
	y = \alpha_1 x + \alpha_0.
	\end{equation}
	
	Se tiene que $\alpha_0$ y $\alpha_1$ vienen dados por este sistema de ecuaciones lineales:
	
	\begin{equation}
	\begin{cases}
	\begin{array}{ccccc}
	\left(\displaystyle\sum_{i=1}^{n} x_i^2\right) \alpha_1 & + & \left(\displaystyle\sum_{i=1}^{n} x_i\right) \alpha_0 & = & \displaystyle \sum_{i=1}^{n} x_i y_i \\
	\left(\displaystyle \sum_{i=1}^{n} x_i\right) \alpha_1 & + & n \alpha_0 & = & \displaystyle \sum_{i=1}^{n} y_i
	\end{array}
	\end{cases}
	\end{equation}
	donde $n$ es el número total de datos.
\end{frame}


\begin{frame}{Linealización de modelos no lineales}
	\begin{table}[htbp!]
		\centering
		\caption{Linealización de distintos modelos}
		\label{tbl:linealizacion}
		\resizebox{\textwidth}{!}{%
			\begin{tabular}{@{}ccc@{}}
				\toprule
				$y = f(x)$                             & \begin{tabular}[c]{@{}c@{}}Forma linealizada\\ $y = \alpha_1 x + \alpha_0$\end{tabular} & \begin{tabular}[c]{@{}c@{}}Cambio de variables\\ y constantes\end{tabular} \\ \midrule
				$y = \dfrac{\alpha_1}{x} + \alpha_0$   & $y = \alpha_1 \dfrac{1}{x} + \alpha_0$                                                  & $X=\dfrac{1}{x}$; $Y=y$                                                    \\
				\addlinespace[1em]
				$y = \dfrac{1}{\alpha_1 x + \alpha_0}$ & $\dfrac{1}{y} = \alpha_1 x + \alpha_0$                                                  & $Y=\dfrac{1}{y}$; $X=x$                                                    \\
				\addlinespace[1em]
				$y = \alpha_1 \log x + \alpha_0$       & $y = \alpha_1 \log x + \alpha_0$                                                        & $Y = y$; $X = \log x$                                                      \\
				\addlinespace[1em]
				$y = \alpha_1 e^{\alpha_0 x}$          & $\log y = \log \alpha_1 + \alpha_2 \log x$                                              & $Y = \log y$; $X = \log x$; $\alpha_1 = \log \alpha_1$                     \\
				\addlinespace[1em]
				$y = (\alpha_0 + \alpha_1 x)^2$        & $\sqrt{y} = \alpha_0 + \alpha_1 x$                                                      & $Y = \sqrt{y}$; $X=x$                                                      \\ \bottomrule
			\end{tabular}
		}
	\end{table} 
\end{frame}

\end{document}