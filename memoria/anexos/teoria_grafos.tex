\chapter{Teoría de grafos}

La teoría de grafos es la rama de las matemáticas que estudia los grafos, objetos matemáticos que constan de dos elementos: los nodos o vértices y las aristas.\\

A continuación, introduciremos el concepto de grafo, algunas operaciones que se pueden realizar con ellos y algunos de los tipos de grafos que utilizaremos a lo largo de la memoria.

\begin{defi}
Un grafo (o grafo no dirigido) es un par $G = (V,E)$ de conjuntos que satisfacen que $E \subseteq V^2$ y $V \cap E = \emptyset$. Los elementos de $V$ se denominan vértices (o nodos) del grafo $G$ y los elementos de $E$ se denominan arcos (o aristas). Una arista entre los véctices $x, y \in V$ se denota como $xy$ o $yx \in E$.
\end{defi}

La forma usual de representar un grafo es dibujar un punto (o círculo) por cada vértice y unir dos de estos dos puntos (o círculos) con una línea para formar un arco. Cómo estén dibujados los vértices y los arcos es irrelevante. sólo importa qué pares de nodos forman una arista y cuáles no.

\begin{ejemplo}

La Figura \ref{fig:grafo} muestra la representación gráfica de un grafo. Matemáticamente, el grafo es el par $(V, E)$ donde

\[ V = \{A, B, C, D\} \]
\[ E = \{\{A,B\},\{A,C\},\{B,C\},\{C,D\} \} \]

\begin{figure}[htb]
\centering
\ejemplografo
\caption{Ejemplo de grafo}
\label{fig:grafo}
\end{figure}

\end{ejemplo}

\begin{defi}
Se llama orden de un grafo $G$ al número de vértices de dicho grafo. Se denota como $|G|$.\\
Un grafo $G$ se dice que es finito si $|G| < \infty$. Si $|G| = \infty$ se dice que el grafo $G$ es infinito.
\end{defi}

\begin{ejemplo}
El grafo de la Figura \ref{fig:grafo} es un grafo finito, puesto que el número de vértices del grafo es $4$.
\end{ejemplo}

\begin{defi}
Dos vértices $x,y \in V$ del grafo $G = (V,E)$ se dicen adyacentes si existe una arista entre $x$ e $y$ (o $xy \in E$).
\end{defi}

\begin{defi}
Un grafo se dice completo si todos sus vértices son adyacentes.
\end{defi}

\begin{ejemplo}
El grafo de la Figura \ref{fig:grafo_completo} es completo ya que todos sus vértices son adyacentes. En efecto, el vértice $A$ tiene una arista que lo une con los nodos $B$, $C$ y $D$. De la misma forma, se comprueba para los vértices $B$, $C$ y $D$.

\begin{figure}[htb]
\centering
\ejemplografocompleto
\caption{Ejemplo de grafo completo}
\label{fig:grafo_completo}
\end{figure}

\end{ejemplo}

\begin{defi}
Sean $G = (V,E)$ y $G' = (V',E')$ dos grafos. Decimos que $G$ y $G'$ son isomorfos, y escribimos $G \simeq G'$, si existe una biyección $\phi : V \to V'$ tal que $xy \in E \iff \phi(x)\phi(y) \in E' \ \ \forall x,y \in V$. La aplicación $\phi$ recibe el nombre de isomorfismo. Si $G = G'$, $\phi$ se dice que es un automorfismo. 
\end{defi}

Podemos definir operaciones sobre grafos, como la unión o la intersección.

\begin{defi}
Sean $G = (V,E)$ y $G' = (V',E')$ dos grafos, se definen la unión y la intersección de grafos como

\begin{eqnarray*}
G \cup G' := (V \cup V', E \cup E')\\
G \cap G' := (V \cap V', E \cap E')
\end{eqnarray*}

Si $G \cap G' = \emptyset$, entonces $G$ y $G'$ son disjuntos.
\end{defi}

\begin{ejemplo}
La Figura \ref{fig:union_interseccion_grafo} muestra la unión e intersección de grafos. Si $G$ y $G'$ son respectivamente 

\begin{eqnarray*}
G = (\{A, B, C, D, E\}, \{ \{A,B\}, \{B,C\}, \{B,D\}, \{C,E\}, \{D, E\} \})\\
G' = (\{C, D, E, F\}, \{ \{C,D\}, \{C,E\}, \{D,F\}, \{E,F\} \})
\end{eqnarray*}

Por definición,

\begin{multline*}
G \cup G' = (\{A, B, C, D, E, F\}, \{ \{A,B\}, \{B,C\}, \{B,D\}, \{C,E\}, \{D, E\}, \\
\{C,D\}, \{D,F\},\{E,F\} \})
\end{multline*}
\[G \cap G' = (\{C, E\}, \{\{C,E\}\})\]


\begin{figure}[htb]
\centering
\ejemplounionintersecciongrafo
\caption{Ejemplo de unión de unión e intersección de grafos}
\label{fig:union_interseccion_grafo}
\end{figure}

\end{ejemplo}

\begin{defi}

Sean $G = (V,E)$ y $G' = (V',E')$ dos grafos. Si $V' \subseteq V$ y $E' \subseteq E$, se dice que $G'$ es un subgrafo de $G$ (y $G$ es un supergrafo de $G'$).

\end{defi}

\begin{ejemplo}
La figura \ref*{fig:subgrafo} muestra algunos de los subgrafos de $G= (V,E)$ donde

\[ V  = \{A, B, C, D, E\}\]
\[E = \{ \{A,B\}, \{A,C\}, \{A,E\}, \{B,D\}, \{B, E\},\{C,D\}, \{D,E\},\{C,E\} \} \]

Del mismo modo, $G' = (V', E')$ y $G'' = (V'', E'')$ donde

\[ V'  = \{A, B, C, D\}\]
\[E' = \{ \{A,B\}, \{A,C\}, \{B,D\},\{C,D\} \} \]

\[ V''  = \{A, B, C, D, E\}\]
\[E'' = \{ \{A,B\}, \{A,C\}, \{B,D\},\{C,D\},\{B,E\} \} \]

Se ve claramente que $V' \subseteq V$ y $E' \subseteq E$, por lo que $G'$ es un subgrafo de $G$ (o $G$ es un supergrafo de $G'$). Análogo para $G''$.


\begin{figure}[htb]
\centering
\ejemplosubgrafo
\caption[Ejemplo de subgrafos de un grafo]{Ejemplo de subgrafos de un grafo $G$}
\label{fig:subgrafo}
\end{figure}

\end{ejemplo}

\begin{defi}
Sea $G = (V,E)$ un grafo (no vacío). El grado de un vértice $v \in V$, denotado por $d_G(v) = d(v)$, se define como el número de vértices adyacentes a $v$.\\

Si todos los vértices de $G$ tienen el mismo grado $k$, el grafo $G$ es regular.
\end{defi}

\begin{defi}
Se define el grado medio de un grafo $G = (V,E)$ como el número

\begin{equation}
d(G) = \dfrac{1}{|V|} \sum_{v \in V} d(v)
\end{equation}
\end{defi}

\begin{defi}
Un camino es un grafo no vacío $P = (V, E)$ de la forma

\begin{equation*}
V = \{ x_0,x_1,\dots,x_k\} \quad \quad E = \{ x_0 x_1, x_1 x_2, \dots, x_{k-1}x_k \}
\end{equation*}

donde $x_i \neq x_j \ \forall i \neq j$.\\

Los vértices $x_0$ y $x_k$ se denominan final del camino $P$. Los vértices $x_1, \dots, x_k$ se denominan vértices interiores del camino $P$.\\

El número de aristas del camino se denomina longitud del camino.
\end{defi}

\begin{defi}
Un grafo no vacío $G$ se dice conexo si cualquier par de vértices están unidos por un camino de $G$.
\end{defi}

\begin{defi}
Sea $G = (V,E)$ un grafo. Un subgrafo conexo maximal de $G$ se llama componente conexa de $G$.
\end{defi}

\begin{defi}
Un clique es un conjunto de nodos mutuamente conectados entre sí.
\end{defi}

\begin{ejemplo}
Un triángulo es un clique formado por tres nodos.
\end{ejemplo}

\begin{defi}
Un grafo dirigido (o digrafo) es un par $(V,E)$ de conjuntos disjuntos (de vértices y de aristas) junto con dos funciones $\mathrm{init} : E \to V$ y $\mathrm{ter} : E \to V$ que asigna a cada arista $e$ un vértice inicial $\mathrm{init}(e)$ y un vértice terminal $\mathrm{ter}(e)$.\\

La arista $e$ se dice dirigida desde $\mathrm{init}(e)$ hasta $\mathrm{ter}(e)$.\\

Si $\mathrm{init}(e) = \mathrm{ter}(e)$, la arista $e$ se dice que es un bucle.
\end{defi}

\begin{ejemplo}
La Figura \ref*{fig:grafo_dirigido} muestra la representación gráfica un grafo dirigido. Matemáticamente, es el par $(V,E)$ donde

\[V  = \{A, B, C, D\}\]
\[E = \{ \{A,B\}, \{A,C\}, \{C,C\},\{B,C\}, \{C,D\} \} \]

junto con las funciones $\mathrm{init} : E \to V $ y $\mathrm{ter} : E \to V$ definidas de la siguiente manera:

\begin{center}
\begin{tabular}{l|r}
\begin{tabular}{r c c c}
$\mathrm{init}:$ & $E$ & $\to$ & $V$\\
			     & $\{A,B\}$ & $\mapsto$ & $A$\\
			     & $\{A,C\}$ & $\mapsto$ & $A$\\
			     & $\{C,C\}$ & $\mapsto$ & $C$\\
			     & $\{B,C\}$ & $\mapsto$ & $B$\\
			     & $\{C,D\}$ & $\mapsto$ & $C$\\
			      
\end{tabular} &
\begin{tabular}{r c c c}
$\mathrm{ter}:$ & $E$ & $\to$ & $V$\\
			     & $\{A,B\}$ & $\mapsto$ & $B$\\
			     & $\{A,C\}$ & $\mapsto$ & $C$\\
			     & $\{C,C\}$ & $\mapsto$ & $C$\\
			     & $\{B,C\}$ & $\mapsto$ & $C$\\
			     & $\{C,D\}$ & $\mapsto$ & $D$\\
			      
\end{tabular} 
\end{tabular}
\end{center}

Además la arista $\{C,C\}$ es un bucle porque $\mathrm{init}(\{C,C\}) = \mathrm{ter}(\{C,C\}) = C$.

\begin{figure}[h]
\centering
\ejemplografodirigido
\caption{Ejemplo de grafo dirigido con bucle}
\label{fig:grafo_dirigido}
\end{figure}

\end{ejemplo}

\begin{defi}\label{def:orientacion}
Un grafo dirigido $D = (V', E')$ es una orientación de un grafo (no dirigido) $G = (V,E)$ si $V = V'$ y $E = E'$ y $\{\mathrm{init}(e), \mathrm{ter}(e) \} = \{x,y\} \ \forall e = xy \in E$.
\end{defi}

\begin{defi}
Un grafo ponderado es un grafo con una función $\mathrm{w} : E \to \R$, es decir, que $\mathrm{w}$ asocia un número real a cada arista. Esta función recibe el nombre de función peso.
\end{defi}

\begin{ejemplo}
El grafo $G = (V,E)$ con 

\begin{eqnarray}
V = \{A, B, C\}\\
E = \{ \{A,B\}, \{A,C\}, \{B,C\} \}
\end{eqnarray} 

es un grafo. Si le añadimos la función $\mathrm{w} : E \to \R$ definida de la siguiente forma, $G$ es un grafo ponderado (ver Figura \ref{fig:grafo_ponderado}).

\begin{equation*}
\begin{tabular}{r c c c}
$\mathrm{w}:$ & $E$ & $\to$ & $\R$\\
			     & $\{A,B\}$ & $\mapsto$ & $e$\\
			     & $\{A,C\}$ & $\mapsto$ & $5$\\
			     & $\{B,C\}$ & $\mapsto$ & $\pi$
			      
\end{tabular}
\end{equation*}
\begin{figure}[htb]
\centering
\ejemplografoponderado
\caption{Ejemplo de grafo ponderado}
\label{fig:grafo_ponderado}
\end{figure} 
\end{ejemplo}

\begin{defi}
Dado un grafo ponderado $G = (V,E)$ y $\mathrm{w} : E \to \R$, decimos que una arista $e \in E$ incide en el vértice $v \in V$, si $\mathrm{ter}(e) = v$. 
\end{defi}

\begin{defi}
La fuerza de un vértice en un grafo ponderado se define como la suma de todos los pesos de sus aristas incidentes.
\end{defi}

\begin{defi} \label{def:dominancia}
Un grafo de dominancia $G=(V,E)$ es un grafo dirigido tal que para todo $x,y \in V$ se cumple una de las dos condiciones siguientes, pero no ambas simultáneamente:

\begin{itemize}
\item $\mathrm{init}(xy) = x$ \quad y \quad $\mathrm{ter}(xy) = y$
\item $\mathrm{init}(xy) = y$ \quad y \quad $\mathrm{ter}(xy) = x$ 
\end{itemize}
\end{defi}

\begin{ejemplo}
Si consideramos el grafo de la Figura \ref{fig:grafo_dominancia}, vemos que para todo vértice $x,y \in \{A, B, C, D\}$ se cumple alguna de las dos condiciones anteriores, pero no ambas simultáneamente. Por ejemplo, para los vértices $A, D$ se cumple que $\mathrm{init}(\{A,D\}) = A$ y $\mathrm{ter}(\{A,D\}) = D$, pero no se cumple la otra condición. De la misma se comprueban los vértices restantes.   

\begin{figure}[h]
\centering
\ejemplografodominancia
\caption{Ejemplo de grafo de dominancia}
\label{fig:grafo_dominancia}
\end{figure}
\end{ejemplo}

\begin{defi}
Llamamos grafo complementario de $G = (V,E)$, y lo denotamos como $\overline{G}$ al grafo que tiene como conjunto de vértices $V$ y como conjunto de aristas, las aristas que no están unidas de $G$.
\end{defi}

\begin{ejemplo}

La Figura~\ref{fig:grafo_complementario} muestra un grafo $G$ y su correspondiente grafo complementario $\overline{G}$. Las aristas que no están unidas en $G$, sí lo están en $\overline{G}$.

\begin{figure}[h]
\centering
\ejemplografocomplementario
\caption{Ejemplo de grafo complementario}
\label{fig:grafo_complementario}
\end{figure}
\end{ejemplo}