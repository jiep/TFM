\chapter{Introducción}

En la actualidad los seres humanos somos unos grandes generadores de datos: al visitar una página web nuestros hábitos de navegación (tipo de ordenador utilizado, navegador, país, hora, etc) quedan almacenados en los servidores de la página web; al realizar una transacción bancaria, nuestro banco guarda datos acerca del cajero utilizado, a qué hora se ha realizado, desde qué país, etc. Éstos son algunos de los lugares donde son recogidos nuestros datos, aunque existen infinidad de ellos.\\

El objetivo no es almacenar únicamente los datos, sino obtener valor de ellos: en el caso de la página web, se usarán los datos para obtener anuncios o compras (si es una tienda) personalizados, o bloquear tráfico, en caso de que se estuviera produciendo un ataque hacia la página web. En el caso del banco, podría usar los datos para para ofrecer productos personalizados o bloquear transacciones que no se adecuen al comportamiento de un usuario.\\

Para lograr ésto, se necesitan algoritmos que permitan a los ordenadores ``aprender'' a partir de un conjunto de datos de entrada, para posteriormente inferir sobre nuevos datos. De ésto se encarga el Machine Learning o Aprendizaje Automático.\\

A lo largo de la memoria veremos qué es el machine learning, qué tipos hay según el tipo de aprendizaje y qué problemas se pueden resolver usándolo, y los algoritmos de machine learning más utilizados.\\

También veremos otros algoritmos más novedosos: las redes parenclíticas, basadas en teoría de grafos.\\

Con todo ésto, desarrollaremos una aplicación web para analizar distintos conjuntos de datos usando algoritmos clásicos de machine learning y redes parenclíticas y poder comparar entre ambos algoritmos.\\

Por último, utilizaremos la aplicación para la detección de tumores en cáncer de mama.  