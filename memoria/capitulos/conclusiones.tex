\chapter{Conclusiones}
Durante la memoria hemos visto algunos de los algoritmos clásicos de machine learning como los árboles de decisión, Support  Vector Machine y las redes neuronales, y un método novedoso basado en teoría de grafos: las redes parenclíticas.\\

Con todos estos algoritmos, hemos desarrollado una aplicación web con R, Shiny Server y Shiny para el estudio de conjuntos de datos mediante redes parenclíticas y los algoritmos clásicos descritos anteriormente.\\

Por último, hemos aplicado la aplicación al diagnóstico de tumores de cáncer de mama, obteniendo grandes resultados, aunque sin llegar al nivel de los algoritmos clásicos. Además, hemos visto que umbralizando las redes parenclíticas es posible obtener mejores resultados que con los grafos completos.
   
\section{Mejoras y futuro trabajo}

\begin{itemize}
	\item Optimización de los algoritmos de redes parenclíticas.
	\item Mejora en visualización de la aplicación.
\end{itemize}