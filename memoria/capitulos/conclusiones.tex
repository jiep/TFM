\chapter{Conclusiones}\label{cap:conclusiones}

Durante la memoria hemos visto qué es el Machine Learning, qué tipos existen según el tipo de aprendizaje y qué tareas permiten resolver. Para cada uno de los tipos de aprendizaje hemos visto qué es lo que caracteriza a cada uno, qué tareas se pueden resolver y algunos de los algoritmos que se pueden aplicar.\\

Hemos visto que es posible usar la teoría de grafos para desarrollar un nuevo algoritmo de Machine Learning para realizar la tarea de clasificación binaria. Hemos visto cómo generalizar el método a una clasificación multiclase, cómo aplicar distintos métodos de regresión para realizar la clasificación.\\

Hemos desarrollado una aplicación que analiza y compara un conjunto de datos según distintos algoritmos de Machine Learning.\\

Por último, hemos aplicado la aplicación al diagnóstico de tumores de cáncer de mama, obteniendo grandes resultados, aunque sin llegar al nivel de los algoritmos clásicos. Además, hemos visto que umbralizando las redes parenclíticas es posible obtener mejores resultados que con los grafos completos.
   
\section{Mejoras y futuro trabajo}

\begin{itemize}
	\item Optimización de los algoritmos de redes parenclíticas: se deben optimizar los algoritmos de aprendizaje de redes parenclíticas para minimizar la cantidad de memoria empleada y el tiempo de procesamiento para conjuntos de datos muy grandes.
	 
	\item Mejora en visualización de la aplicación: mejora de la interfaz gráfica para conseguir una mejor experiencia de usuario, con la consiguiente migración a HTML 5 y CSS 3 ``nativo''.
	
	\item Nuevos métodos de regresión: adición de nuevos métodos de regresión como la regresión logística o la no lineal, o incluso, algoritmos de Machine Learning que permitan la tarea de regresión, como SVM o redes neuronales.
	
	\item Persistencia de las redes parenclíticas en bases de datos de grafos: almacenamiento persistente de las redes parenclíticas en bases de datos de grafos como OrientDB~\cite{orientdb}, Titan~\cite{titan} o Neo4j~\cite{neo4j} para realizar futuros procesamientos o visualizaciones.
\end{itemize}